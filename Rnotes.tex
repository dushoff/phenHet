\documentclass[12pt]{article}
\begin{document}

\newcommand{\Rx}[1]{\ensuremath{\mathcal{R}_{#1}}}
\newcommand{\emd}{\ensuremath{\omega}}

Define the rate ratio $\rho = \beta/\gamma$, and the “weighted mean degree” (node degree averaged over edge) $\emd = <d^2>/<d> = \delta(1+
\kappa)$.

We then have the initial case reproductive number (expected number of cases caused by a given case):

$$ \Rx{0,c} = \frac{\rho}{\rho+1}(\emd-1). $$

This should make sense. As the disease spreads initially, $\emd - 1$ is the average number of available neighbors (since one neighbor infected the focal node), and $\frac{\rho-1}{\rho}$ is the probability each neighbor will be infected. 

Note that \Rx{0,c} is always positive (zero-edge nodes have weight 0, all other integers are $\geq1$), but can never exceed the threshold of one unless $\emd>2$ (since $\emd-1$ is being multiplied by a probability. This makes sense: if the effective mean number of neighors is two, than the number the average can infect after being infected is capped at one.

Define the instantaneous reproductive number as the counterfactual number of individuals who would be infected by an observed force-of-infection/duration-of-infectiousness combination, assuming no other changes. This suddenly seems complicated in general, but is still going to be simple for constant-duration Markovian infections -- and it gives the value we would like to use for \Rx{\mathit{eff}}\ in our phenHet approaches. We believe that:

$$ \Rx{0,i} = \rho(\emd-2).$$

This is cool, and at least a little bit weird. We already had the idea that we need $\emd>2$ for disease spread. But it seems a little weird that $\emd-2$ comes in so simply. It would be nice to have a good intuition for it. It's also noticeable that when \emd\ is between 1 and 2, which is perfectly plausible, our formula gives a negative value for \Rx{0,i}.

The likely answer to the second weirdness is that -- since the disease cannot spread if $\emd\leq2$ -- the eigenstate under which our formula applies asymptotically evaporates as $\emd\to2$ so that our equation is not expected to apply after that. This currently seems plausible to JD, but could be questioned.

The likely answer to the first weirdness is that, in the spreading eigenstate we're envisioning, each currently infected node has one infector and, on average, one neighbor that \emph{it} has already infected.

So: what's the intuitive reason why this number is 1 (and independent of $\rho$)? Can we get some intuition for how it might change as the disease spreads beyond the eigenstate? If so, we could get a link between \Rx{t,i}\ and \Rx{t,0}.

\end{document}
